\documentclass[12pt]{article}
\usepackage[margin=1.0in]{geometry}
\usepackage{amssymb}	% for \square check-box
\usepackage{hyperref}
\pagestyle{empty}
\newcounter{lastNum}   % allows for interspersing text between questions
\begin{document}
	
	\begin{center}
		\textbf{Data Science Techniques (MAT 339)} \\
		\textbf{Homework 6 - PART I} - - \textbf{HW 6 will be worth 10 points total}\\
	\end{center}
	Submit a \textbf{hard copy }of your work at the beginning of class on Wednesday, February 25th. There is no \textbf{electronic submission }for PART I of this homework. \\


\begin{enumerate}
	
	\item In this question you will read about the basics of a powerful automation tool known as \texttt{make} and install it on your system.
	\begin{enumerate}
		\item Read the \texttt{lec-makefiles.pptx} slides in the \texttt{lec05-automation-shell-scripts} folder of the course repo.  \textbf{Optionally} also read the \textbf{Build Systems} section of this web page: \href{https://missing.csail.mit.edu/2020/metaprogramming/}{https://missing.csail.mit.edu/2020/metaprogramming/}
		\item Install make in your system by following the instructions in \texttt{"Installing make for Git Bash.docx"} in the \textbf{Admin} folder of the course repo.
	\end{enumerate}


	\item In this question you will create and run a very simple \texttt{makefile}.
	\begin{enumerate}
		\item Create a one-line text file named \texttt{name01.dat} that contains one name, e.g. "Bart".
		\item Write a Python file named \texttt{hello.py} that reads in \texttt{name01.dat} and prints a greeting with that person's name, as in "Hello Bart!".
		\item You may already know you can run Python files from the command line if your terminal has Python installed. Earlier in the semester you should have set up \textbf{Anaconda} to run within \textbf{Git Bash} enabling this capability. In the terminal, in the same folder as the two files you just created, type \texttt{python hello.py}.  Include this terminal line and the output as your answer to this part. 
		\item Following the information in the slides you were to read above, create a simple \texttt{makefile} (you can name it \texttt{Makefile} or \texttt{Makefile.print-hello}) that runs \texttt{hello.py} and pipes its output (use \texttt{>}) to a file named \texttt{hello.txt}.  The dependencies of \texttt{hello.txt} in the \texttt{makefile} should be \texttt{name01.dat} and \texttt{hello.py}. Include your \texttt{makefile} and successful execution of it as your answer to this part.
		\item Without changing \texttt{name01.dat}, run the \texttt{makefile} again. Supply the output as your answer to this part.
		\item Change the contents of \texttt{name01.dat} and run the \texttt{makefile}. Supply the output as your answer to this part. 
	\end{enumerate}
	
	\item THIS QUESTION IS OPTIONAL. In order to automate \LaTeX\  file production using \texttt{make}, install \textbf{MiKTeX}, and optionally a \LaTeX\  editor like \textbf{TeXstudio}. Links are available through a Google search.
\end{enumerate}

\end{document}