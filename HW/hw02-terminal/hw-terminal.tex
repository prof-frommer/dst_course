\documentclass[12pt]{article}
\usepackage[margin=1.0in]{geometry}
\usepackage{amssymb}	% for \square check-box
\usepackage{hyperref}
\pagestyle{empty}
\newcounter{lastNum}   % allows for interspersing text between questions
\begin{document}

\begin{center}
	\textbf{Data Science Techniques (MAT 339)} \\
	\textbf{Homework 2} - - \textbf{10 Points} \\
\end{center}	
Submit a \textbf{hard copy }of your work at the beginning of class on Wednesday, January 28th.  There is no \textbf{electronic submission }for this homework.

\begin{enumerate}
	\item Download the free book \textit{The Linux Command Line} from \url{https://linuxcommand.org/tlcl.php}. Although we are not using Linux, much of the content of this book applies to us in Git Bash.
	\begin{enumerate}
		\item 	Read pages xvi to xviii of the \textbf{Introduction}. 
		\item   Spend at least another 10-15 minute skimming the book and jot down a few pieces of information you learned as your answer to this question.
	\end{enumerate}

	
	\item On your computer find (or create) a folder with numerous files in it (and possibly folders).  Use an \texttt{ls} command with some of its options (use \texttt{-l} and others) to list the files. Then use the approach shown in class to redirect the listing to a file (name it \texttt{file-listing.txt} or similar). For your answer to this question include:
	\begin{enumerate}
		\item your line of code with the redirect
		\item a screen shot of the resulting text file open in Notepad or similar
	\end{enumerate} 
	
	\item Jake VanderPlas' \textit{Python Data Science Handbook} is available from a GitHub repo at \url{https://github.com/jakevdp/PythonDataScienceHandbook}. Start by cloning this repo to your machine. (Note it is around 80 Mb, so if you are low on space, find a different smaller repo online to use.) Using shell commands covered in class (including \texttt{ls} with options), find all Jupyter files in this repo and list their names with details in decreasing order of file size.
	
	\item Go a folder on your machine with regular activity of files being saved there, such as your \textbf{Downloads} or \textbf{Documents} folder.  
	\begin{enumerate}
		\item Look up the \texttt{-mtime} option of the \texttt{find} command in the \textit{The Linux Command Line} book you downloaded at the beginning of this homework and read about it.  What page was it on?  
		\item Use the \texttt{-mtime} option to list files in the folder modified within some recent amount of time, enough to return some but not too many results. Do not submit this result.
		\item Now repeat the previous step but pipe it to the \texttt{tr} command changing the letters (and possibly numbers as well) to something that would cloak them, e.g. change all letters to B's.  There are examples of the usage of \texttt{tr} in the slides from Day 2 and in the book. Include your command for this part and a screen shot (or pasted text) of the result in your homework submission.
	\end{enumerate}

\end{enumerate}



\end{document}