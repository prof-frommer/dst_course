\documentclass[11pt]{article}
\usepackage[margin=1.0in]{geometry}
\usepackage{amssymb}	% for \square check-box
\usepackage{hyperref}
\pagestyle{empty}
\newcounter{lastNum}   % allows for interspersing text between questions
\usepackage{listings}  
\lstset{breaklines=true,  basicstyle=\ttfamily\small}

\begin{document}

\begin{center}
	\textbf{Data Science Techniques (MAT 339)} \\
	\textbf{Homework 2} - - \textbf{10 Points} - - \textbf{SOLUTIONS} \\
\end{center}	
Submit a \textbf{hard copy }of your work at the beginning of class on Wednesday, January 28th.  There is no \textbf{electronic submission }for this homework.

\begin{enumerate}
	\item Download the free book \textit{The Linux Command Line} from \url{https://linuxcommand.org/tlcl.php}. Although we are not using Linux, much of the content of this book applies to us in Git Bash.
	\begin{enumerate}
		\item 	Read pages xvi to xviii of the \textbf{Introduction}. 
		\item   Spend at least another 10-15 minute skimming the book and jot down a few pieces of information you learned as your answer to this question.
	\end{enumerate}

	
	\item On your computer find (or create) a folder with numerous files in it (and possibly folders).  Use an \texttt{ls} command with some of its options (use \texttt{-l} and others) to list the files. Then use the approach shown in class to redirect the listing to a file (name it \texttt{file-listing.txt} or similar). For your answer to this question include:
	\begin{enumerate}
		\item your line of code with the redirect \\
		\textbf{SOLUTION: } \texttt{ls -l > filelisting.txt} 				
		\item a screen shot of the resulting text file open in Notepad or similar \\
		\textbf{SOLUTION: } a snippet of the output file \texttt{file-listing.txt}:
	\begin{lstlisting} 
-rw-r--r-- 1 IFrommer 1049089  23721 Feb 12  2024 2024 ORDA catalog updates.docx
-rw-r--r-- 1 IFrommer 1049089  16902 Apr  9  2024 202425 HPE Curriculum Update Advisor Guidance.docx
drwxr-xr-x 1 IFrommer 1049089      0 Dec 30  2024 Admin/
drwxr-xr-x 1 IFrommer 1049089      0 Jul 14  2023 Apps/
drwxr-xr-x 1 IFrommer 1049089      0 Jan  4  2023 Attachments/
-rw-r--r-- 1 IFrommer 1049089  14239 May 20  2025 Awards luncheon.docx
-rw-r--r-- 1 IFrommer 1049089  10007 Mar  7  2024 Book1.xlsx
-rw-r--r-- 1 IFrommer 1049089  10281 May 13  2025 Book2.xlsx
-rw-r--r-- 1 IFrommer 1049089  92966 Feb  7  2025 CAP question writing.docx
-rw-r--r-- 1 IFrommer 1049089  11633 Jan 15  2024 Calc 3111-101 Section.xlsx
	\end{lstlisting}		
	\end{enumerate} 
	
	\item Jake VanderPlas' \textit{Python Data Science Handbook} is available from a GitHub repo at \url{https://github.com/jakevdp/PythonDataScienceHandbook}. Start by cloning this repo to your machine. (Note it is around 80 Mb, so if you are low on space, find a different smaller repo online to use.) Using shell commands covered in class (including \texttt{ls} with options), find all Jupyter files in this repo and list their names with details in decreasing order of file size.
	
	\textbf{SOLUTION:}  \\
	From the root directory of the repo: \\
	\texttt{find . -name *ipynb -print0 | xargs -0 ls -lhS | less} \\
	-S to sort by decreasing size, -H for human-readable sizes.  Here are the beginnings of the results:
	\begin{lstlisting}
-rw-r--r-- 1 IFrommer 1049089  2.7M Feb  5 12:52 ./notebooks_v1/04.13-Geographic-Data-With-Basemap.ipynb
-rw-r--r-- 1 IFrommer 1049089  2.1M Feb  5 12:52 ./notebooks_v1/05.11-K-Means.ipynb
-rw-r--r-- 1 IFrommer 1049089  1.8M Feb  5 12:52 ./notebooks_v1/06.00-Figure-Code.ipynb
-rw-r--r-- 1 IFrommer 1049089  1.7M Feb  5 12:52 ./notebooks/05.11-K-Means.ipynb
-rw-r--r-- 1 IFrommer 1049089  1.7M Feb  5 12:52 ./notebooks/06.00-Figure-Code.ipynb
-rw-r--r-- 1 IFrommer 1049089  1.5M Feb  5 12:52 ./notebooks_v1/05.10-Manifold-Learning.ipynb
-rw-r--r-- 1 IFrommer 1049089  1.1M Feb  5 12:52 ./notebooks_v1/05.12-Gaussian-Mixtures.ipynb
-rw-r--r-- 1 IFrommer 1049089  1.1M Feb  5 12:52 ./notebooks_v1/04.14-Visualization-With-Seaborn.ipynb
-rw-r--r-- 1 IFrommer 1049089 1010K Feb  5 12:52 ./notebooks_v1/05.07-Support-Vector-Machines.ipynb
-rw-r--r-- 1 IFrommer 1049089  953K Feb  5 12:52 ./notebooks/05.10-Manifold-Learning.ipynb
	\end{lstlisting}
		
	
	\item Go a folder on your machine with regular activity of files being saved there, such as your \textbf{Downloads} or \textbf{Documents} folder.  
	\begin{enumerate}
		\item Look up the \texttt{-mtime} option of the \texttt{find} command in the \textit{The Linux Command Line} book you downloaded at the beginning of this homework and read about it.  What page was it on?  \\
		\textbf{SOLUTION:} p237 of the book (p261 of the pdf)
		\item Use the \texttt{-mtime} option to list files in the folder modified within some recent amount of time, enough to return some but not too many results. Do not submit this result. \\
		\textbf{SOLUTION:} \texttt{find . -type f -mtime -1}  (using -1 to capture files modified within the past 1 day)
		
		\item Now repeat the previous step but pipe it to the \texttt{tr} command changing the letters (and possibly numbers as well) to something that would cloak them, e.g. change all letters to B's.  There are examples of the usage of \texttt{tr} in the slides from Day 2 and in the book. Include your command for this part and a screen shot (or pasted text) of the result in your homework submission. \\
		\textbf{SOLUTION:} \texttt{find . -type f -mtime -1 | tr [a-z] 7}
\begin{lstlisting}
./.777/777777
./.777/FETCH_HEAD
./.777/7777/7777/7777777/777777/HEAD
./.777/7777/7777777/777777/HEAD
./_7777777777/HW/7702-77777777/77-77777777-7777777-77777.777
./_7777777777/HW/7702-77777777/77-77777777-7777777-77777.777
./_7777777777/HW/7702-77777777/77-77777777-7777777-77777.777
./_7777777777/HW/7702-77777777/77-77777777-7777777-77777.777
./_7777777777/HW/7702-77777777/77-77777777-7777777-77777.7777777.77
./_7777777777/HW/7702-77777777/77-77777777-7777777-77777.777
./_7777777777/HW/7702-77777777/777/1.777
./_7777777777/HW/7702-77777777/777/2.777
\end{lstlisting}		

	\end{enumerate}

\end{enumerate}



\end{document}